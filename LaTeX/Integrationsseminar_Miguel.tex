Gliederung:
    Hardware
        Raspberry Pi 3
            SOC, I/O, OS
        Raspberry Pi 4
            Unterschiede zu Pi 3
        SPI
        
        
    


\chapter{Raspberry Pi 3}
Der Raspberry Pi ist ein vollständiger Computer in der Größe einer Kreditkarte. Er wird von der nicht profitorientierten Raspberry Pi Foundation entwickelt und in Großbritannien hergestellt. Er besitzt keinen aufgelöteten Speicher, sondern bootet direkt von einer MicroSD-Karte. Die Raspberry Pi Foundation entwickelt auch das offizielle Betriebssystem Pi OS auf Basis von Linux (Debian).

Die Zweite Version des Raspberry Pi (2 Mod. B v1.2) hat 1GB RAM und als CPU vier ARM Cortex-A53-Kerne, welche die ARMv8-A-Mikroarchitektur implementieren. Für dieses Modell gibt es aktuell keine offizielle 64-bit-Version von Pi OS, was die Kompatibilität mit bestimmter Software einschränkt.
Die Schnittstellen umfassen MicroUSB für die Stromversorgung, HDMI für die Bildausgabe, 3,5mm Klinke für Audio, RJ-45 (Ethernet), 4x USB 2.0, 26 GPIO-Pins und zwei Flachband-Header für den offiziellen Touchscreen und die offizielle Kamera.

Die vierte Version (4 Mod. B) hat eine schnellere CPU mit vier ARM Cortex-A72-Kernen, 4GB RAM, fest verbautes Wi-fi und bluetooth sowie zwei Micro-HDMI-Buchsen. Die Form ist gleich geblieben.

\chapter{Intel Movidius Myriad X}
Der Myriad X der Intel-Tochterfirma Movidius ist eine Vision Processing Unit (VPU), die auf IoT-Anwendungen spezialisiert ist. Sie unterstützt bis zu 8 Kameras mit einer Auflösung von 4k. Sie kann Aufgaben wie Stereosicht und Bildverarbeitung sehr effizient ausführen und ist dabei nicht auf Datentransfers zu externem Speicher angewiesen.

Zusätzlich ist sie mit einer Neural Compute Engine ausgestattet, die ein schnelles und Energieeffizientes Ausführen von Inferenz auf neuralen Netzen ermöglicht. Movidius gibt eine Performance von 1 TOPS (1 Billion Operationen pro Sekunde) an.

\chapter{Luxonis OAK-D}
Die Luxonis OAK-D ist eine IoT-Kamera, die einen RGB-Sensor und ein Paar Stereosensoren hat. Der Stereosensor nutzt einen Sony IMX378-Sensor und kann Video in 4k aufnehmen. Das Stereopaar nutzt Omnivision OV9282-Sensoren mit einer Auflösung von 1280x800.
Die Sensoren sind direkt mit der integrierten Myriad X-VPU verbunden. Stromversorgung und Datentransfer erfolgen über USB-C, wobei sowohl USB 2.0 als auch USB 3.0 unterstützt werden.

Die Kamera wird über die DepthAI-Platform und deren Python-API gesteuert. Hierbei kann der Entwickler selbst entscheiden, welche Sensoren und Funktionen der VPU er in welchem Ausmaß nutzt.

\chapter{RC522 RFID-Modul}

Das RC552-Modul von NXP Semiconductors ist ein RFID-Transponder, der das 13.56Mhz ISM-Band nutzt und dadurch keine Probleme mit anderen Funkverbindungen verursacht. Es kann über UART, I²C oder SPI kommunizieren und wird mit 3.3V betrieben, passt also auf die GPIO-Pins des Raspberry Pi. Eine Besonderheit ist der Interrupt-Pin, mit dem ein externes Gerät aufgeweckt werden kann, sobald ein RFID-Tag erkannt wird. Es wird mit zwei RFID-Tags geliefert, die beliebig beschrieben und ausgelesen werden können. 

\chapter{SSH}

Secure Shell ist ein Netzwerkprotokoll, mit dem verschlüselt auf entfernte Rechner zugegriffen werden kann. Der entfernte Rechner muss dafür direkt erreichbar sein (über ein lokales Netzwerk oder eine Portweiterleitung).
SSH ist ein Client-Server-Protokoll, wobei der entfernte Rechner der Server und der lokale Rechner der Client ist. Neben klassischen textbasierten Shells können mit SSH auch verschlüsselte Dateiübertragungen (per SFTP) und grafische Anwendungen (per X11-Forwarding) realisiert werden.

\chapter{Evaluation der Pipeline}

Die Kennzeichenerkennungspipeline besteht aus folgenden Schritten:
\begin{itemize}
    \item[Laden der kurz zuvor gespeicherten Datei]
    \item[Preprocessing]
    \begin{itemize}
        \item[Umwandlung zu Graustufen]
        \item[Bilateraler Filter]
        \item[Canny-Algorithmus zu Kantenfindung]
        \item[Douglas-Peucker-Algorithmus zur Erkennung von Rechtecken]
        \item[Cropping]
        \item[OCR mit vorgefertigter Lösung]
    \end{itemize}
\end{itemize}

Es war geplant, an den Anfang dieser Pipeline noch ein neuronales Netz zu hängen, mit dem eine erste Eingrenzung des Kennzeichens erfolgen sollte. Dieser Schritt wurde jedoch aufgrund von technischen Problemen entfernt.

Als OCR-Lösung war EasyOCR geplant, weil es trotz geringem Ressourcenaufwand gute Ergebnisse liefert. Dieser Ansatz scheiterte, weil EasyOCR von Torch abhängt, Torch ein 64bit-Betriebssystem voraussetzt und Pi OS für den Raspberry Pi 2 nur als 32bit-Version verfügbar ist.
Die Alternative ist Tesseract, eine OCR-Engine, die ursprünglich von Hewlett-Packard entwickelt wurde. Seit 2005 ist es Open-Source und von 2006 bis 2018 wurde es von Google weiterentwickelt. https://tesseract-ocr.github.io/docs/tesseracticdar2007.pdf

\chapter{Fuzzy Matching}
Beide OCR-Lösungen liefern ungenaue Ergebnisse. Die Plaketten zwischen den Blöcken der Kennzeichen werden oft als Zeichen interpretiert. Genauso wird zwischen den Blöcken manchmal ein Leerzeichen erkannt und manchmal nicht.
Bei einem genauen Vergleich dieser Ergebnisse mit einer Liste von erlaubten Kennzeichen würde daher sehr selten ein Treffer auftreten.

Aus diesem Grund wird stattdessen Fuzzy Matching verwendet. Fuzzy matching 