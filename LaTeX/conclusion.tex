% !TEX root =  master.tex
\chapter{Fazit}

\nocite{*}

Dieses Kapitel enthält die Zusammenfassung des Projektberichts mit einem kurzen Fazit und Ausblick, der weitere Verbesserungen vorschlägt.

\section{Zusammenfassung}
Zusammenfassend lässt sich feststellen, dass die Implementierung der Module gut funktioniert hat und diese zuverlässig funktionieren.

Dies ist nicht zuletzt der weiten Verbreitung des Raspberry Pi und der verwendeten Bauteile unter Hobbybastlern geschuldet, die in der Szene zahlreiche und umfangreiche Tutorials veröffentlichten.

Als weitaus problematischer als gedacht erwies sich jedoch die Umsetzung der Kennzeichenerkennung sowohl durch die Luxonis-Kamera als auch die Alternativlösung. Bei Ersterer scheinen die Schnittstellen und Softwareversionen noch nicht restlos aufeinander abgestimmt zu sein und es finden sich wenige Artikel und Hilfestellungen im Internet, was beides auch an dem erst kürzlich erfolgten Markteintritt liegen kann.

Die schlussendlich verwendete geometrische Lösung funktioniert prinzipiell, jedoch war die Erkennungsrate mit bestenfalls 26\% und die Programmablaufdauer knapp 30 Sekunden unter bzw. über den Erwartungen. 

Da die Kennzeichenerkennung aber ein elementarer Kernbestandteil des Projekts ist, kommen wir zu dem Schluss, dass die technische Umsetzung des Projekts prinzipiell machbar ist, die Reaktionszeit und die Fehlerquote jedoch zu hoch ist für einen sinnvollen Betrieb. Bei über Zwei Minuten Wartezeit haben die meisten Personen wohl ihr Garagentor schneller manuell Geöffnet. Daher bedarf es für den Alltagseinsatz einer Weiterentwicklung.

\section{Ausblick}
Die Nutzung eines 64bit-Betriebssystems würde die Nutzung von easyOCR ermöglichen, was in der Evaluation zu deutlich besserer Performance führte.
Mit zusätzlichem Aufwand ist denkbar, die ursprünglich geplante Eingrenzung des Kennzeichens durch ein neuronales Netz, welches auf der Myriad X VPU läuft, nachträglich zu verwenden und die Schwächen der aktuellen geometrischen Lösung zu umgehen. Alternativ könnte auch kommerzielle Software zur Kennzeichenerkennung eingekauft werden.

Wäre das Performance-Problem gelöst, bestünden vielfältige Möglichkeiten das System zu erweitern oder in anderen Use-Cases zu verwenden. 
Auch eine kommerzielle Nutzung bei Tiefgaragen sowie Parkhäusern oder Mautsytemen wäre möglich. Dies ermöglicht potentiellen Betreibern von diesem System, die Parkdauer oder Fahrtstrecke des Kundens direkt zu berechnen, da das Loggen der Ein- und Ausfahrten in diesem Anwendungsbeispiel diese Information bereits mitliefert. 

Eine Weiterentwicklung des RFID-Zugangs und Kombination mit einer Zeitsteuerung wäre ebenfalls möglich, so könnte der Postbote mit seinem Chip zwischen 8 und 12 Uhr Zutritt erhalten oder der Gärtner zwischen 9 und 17 Uhr.
Auch wäre es möglich weitere IoT-Geräte anzuschließen und beispielsweise bei der Einfahrt schonmal die Lichter im Haus anzuschalten, einen Kaffee zu kochen oder das Badezimmer anzuheizen oder bei der Ausfahrt alle Lichter auszuschalten und die Alarmanlage scharf zu schalten. Der Integration ins Smart Home sind keine Grenzen gesetzt. SmartGarage kann also nicht nur das Garagentor, sondern auch die Tür zu vielen spannenden und nützlichen Anwendungsmöglichkeiten öffnen.


%Hinsichtlich des RFID-Transponders besteht hierbei neben Chipkarten und RFID-Karten auch die Möglichkeit sich einen NFC-Chip beziehungsweise RFID-Chip mittels Implantat unter die Haut einsetzen zu lassen. Dies bietet den Vorteil, dass man immer Zugriff auf seine Garage hat, ohne einen externen Gegenstand, den man verlieren kann, bei sich haben zu müssen.%
%Wild


