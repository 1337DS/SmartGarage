% !TEX root =  master.tex
\chapter{Zusammenfassung}

\nocite{*}

Dieses Kapitel enthält die Zusammenfassung der Arbeit mit Fazit und Ausblick.

\section{Fazit}

super toll 100 Punkte

\section{Ausblick}

Die zukünftigen Verbesserungsmöglichkeiten sind Vielfältig, ebenso so wie die erweiterungen der Anwendungsmöglichkeiten.

Zunächst bietet eine Hardware mit einer besseren Rechenleistung eine bessere Performance, mit der man die Reaktionszeit der Kennzeichenerkennung deutlich verringern könnte. Auch eine Optimierung des Codes wäre möglich. Da es sich nur um ein zeitlich begrenztes Projekt handelte mit dem Ziel ein Proof of Concept zu erstellen, sind hier noch nicht alle mittel und Möglichkeiten ausgeschöpft worden. Denkbar wäre z.B. eine Vorschaltung eines Bewegungsmelders, der dann den Start der Erkennungsschleife auslöst und verhindert, dass der Rasberry Pi zum Zeitpunkt der Einfahrt gerade mit der Auswertung des letzten Bildes beschäftigt ist.

Auch abseits des Smart Home-Bereichs gibt es Use Cases für die Kennzeichenerkennung, wie beispielsweise die Nutzung in kommerziellen Tiefgaragen, an Mautsystemen oder in Fahndungssystemen der Polizei.

Nicht zuletzt 
---------------------------------

Das Projekt SmartGarage bietet in vielen Punkten Potential und Ausblick auf weitere Features. Besonders der Ausblick auf die kommerzielle Nutzung des Systems ist vielversprechend. Grund hierfür ist, dass es aktuell noch kein Gesamtpaket kommerziell zu kaufen gibt, das alle Features, die im Projekt implementiert sind, enthält. Auch die kommerzielle Nutzung im Öffentlichen Raum ist möglich. So beispielsweise bei Tiefgaragen sowie Parkhäusern. Hierbei besteht die Möglichkeit das Projekt in einer abgewandten Form kommerziell Nutzbar zu machen. Hierbei ist es beispielsweise denkbar, die Weboberfläche so umzugestalten, dass dort Parkplätze bereits vorab am Handy oder Tablet gebucht werden können und der Kunde sein Nummernschild angibt. Ist der Kunde dann vor Ort kann er aufgrund der Nummernschilderkennung einfach und ohne weiteren Aufwand in das Parkhaus oder die Tiefgarage fahren und hat bereits einen reservierten Parkplatz. Dies ermöglicht potentiellen Betreibern von diesem System, die Parkdauer des Kundens direkt zu berechnen, da das Loggen der Ein- und Ausfahrten in diesem Anwendungsbeispiel diese Information bereits mitliefert. 

Einen weiteren Nutzen für die Öffentliche Sicherheit kann das Projekt mit einer Anbindung an die Polizei bieten. Durch die Nummernschilderkennung wird bei Einfahrt das Nummernschild erkannt und es ist im System ersichtlich, welches Auto in welcher Garage geparkt hat. Dies macht es wie im Anwendungsbeispiel der Tiefgaragen und Parkhäuser möglich, dass die Polizei direkt nachverfolgen kann, wo ein Nummernschild zuletzt geparkt hat. Wird so beispielsweise ein Verkehrsunfall mit Fahrerflucht gemeldet, bei dem der flüchtige Fahrer danach zu einem Kaufhaus fährt, welches über ein Parkhaus mit einem solchen System verfügt, kann die Polizei den flüchtigen Fahrer nach Einfahrt in das Parkhaus direkt lokalisieren. Dies ist auch im Falle von Privatgaragen möglich, sollten diese auch die Information über Ein- und Ausfahrt mit der Polizei teilen. Im Falle dessen bietet das System hier auch den Vorteil, dass die Zustellung von Strafzetteln vereinfacht wird, da hier die Strafzettel immer an die richtige Adresse gesendet werden.

Potential zur Verbesserung weist das Projekt im Punkt Sicherheit auf. Hierbei ist die Rede von sowohl Sicherheit im Sinne von Schutz vor einem Einbruch als auch Sicherheit im Sinne von Unfallvermeidung. Die Vermeidung von Unfällen spielt vor allem eine Rolle, falls es sich um ein Garagentor handelt, welches sich nach außen öffnet. Hierbei ist es sinnvoll, auch eine Abstandsmessung außerhalb durchzuführen, um das Öffnen der Garage zu blockieren, falls sich etwas vor dem Tor befindet. Hierbei kann es sonst zu Kollisionen zwischen dem Tor der Garage und beispielsweise einem Auto kommen. Kommt es jedoch vor, dass vor der Garage Passanten laufen könnten, könnte es hierbei auch zu schwereren Unfällen kommen, im Falle dass das Garagentor mit einem vorbeifahrenden Fahrrad kollidiert.
Die Sicherheit vor einem Einbruch lässt sich auch durch verschiedene Methoden noch optimieren. Eine Möglichkeit ist hierbei beispielsweise die Implementierung einer Sperre, die das Garagentor mittels Nummernschilderkennung nicht öffnet, sollte sich ein Auto bereits in der Garage befinden. Eine derartige Implementierung schützt nicht nur vor eventuell gefälschten Nummernschildern, sondern auch vor False-Positives im System, wodurch die Garage fehlerhafterweise bei der Einfahrt eines Kennzeichens, welches nicht in der Whitelist steht, geöffnet werden kann. 
