% !TEX root =  master.tex
\chapter*{Kurzfassung (Abstract)}
\addcontentsline{toc}{chapter}{Kurzfassung (Abstract)}

Dieser Projektbericht behandelt die Umsetzung des Projektes \textit{SmartGarage} einer Gruppe von Wirtschaftsinformatikstudenten an der dualen Hochschule Baden-Württemberg - Mannheim.
Ziel des Projektes ist die Entwicklung einer smarte Garagentorsteuerung, die Kennzeichen heranfahrender Fahrzeuge erkennt, auswertet und gegebenenfalls das Garagentor öffnet. Zusätzlich werden weitere Öffnungsmöglichkeiten wie durch einen Sprachassistenten, RFID, und eine Weboberfläche implementiert sowie eine Logging-Funktion. Die technischen Grundlagen der verwendeten Bauteile und Software werden dargestellt sowie die Probleme und Lösungswege bei der Umsetzung erläutert und begründet. \newline 
Der Bericht kommt zu dem Schluss, dass bis auf die Kennzeichenerkennung sämtliche Öffnungsmöglichkeiten fehlerfrei und zuverlässig eingerichtet und angewandt werden könnnen. Die Kennzeichenerkennung selbst funktioniert zwar prinzipiell und in einzelnen Tests, jedoch ist die Gesamterkennungsrate für den Praktischen Betrieb zu gering. Es gibt Anhaltspunkte dafür, dass sich das Problem mit einiger Nacharbeit oder besserer Hardware vermutlich lösen lässt.

\section*{Abstract (english)}
This project report contains the implementation of the project \textit{SmartGarage} of a group of BIS-students of DHBW Mannheim.
Goal of the project is to develop a smart garage door opener, that will recognize the licence place of any incoming car, evaluate the letters and opens the door for authorized vehicles. 
Additionally other useful opening options have been implemented, like opening with a digital virtual assistant, Website, and RFID-Sensor as well as a logging functionality. The fundamental operating principles of the used modules and software are explained and the encountered problems as well as solutions are explained and described. The report comes to the conclusion, that apart from the licence plate recognition, all parts could be implemented and worked well. The licence plate recognition itself works sporadically, but the error rate and runtime is very high. Hence the authors deem the product unfit for practical and commercial use. There are, however, indications that the problems might be solved with more softwareengineering work or better hardware.
