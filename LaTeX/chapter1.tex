% !TEX root =  master.tex
\chapter{Grundlagen}
\section{Konzeption}
Als Ausgangslage dient in diesem Projekt eine handelsübliches elektrisch öffenbares Garagentor, das mit einer Fernbedienung gesteuert wird. Dabei wird davon ausgegangen, dass es sich um eine 1-Kanal Steuerung handelt, also dass ein einmaliges Drücken eines Knopfes der Fernbedienung das Garagentor öffnet und ein weiteres Betätigen desselben Knopfes das Garagentor wieder schließt. Weiterhin wird davon ausgegangen, dass das Garagentor beim Auftreffen auf ein Hindernis automatisch Reversiert. Dies ist sogar, wie durch ein Gerichtsurteil des OLG Frankfurt von 2015 festgestellt, gesetzlich vorgeschrieben.  https://openjur.de/u/775737.html

Neben der automatisierten Öffnung durch die Kennzeichenerkennung sollten folgende Features vorhanden sein:

- Amazon Alexa oder Google Home Integration, um das Garagentor auch als "Fußgänger" und ferngesteuert öffnen zu können (um Gartengeräte zu entnehmen oder den Postboten ein Paket abstellen zu lassen)

- Logging der Ein- und Ausfahrenden Fahrzeuge, um das Produkt eventuell später auch für kommerzielle Parkhäuser und Tiefgaragen nutzen zu können

Folgende Features wurden als optional festgehalten und deren Implementierung vom Projektverlauf abhängig gemacht:

- Lichtschranke zur Erkennung ob die Garage bereits belegt ist. In diesem Fall soll das Tor nicht geöffnet werden und es einem anderen Fahrzeug, das ebenfalls einfahrtsberechtigt ist, ermöglicht weden VOR der Garage zu parkieren ohne ständig durch das sichtbare Kennzeichen die automatische Öffnung auszulösen.

- Öffnung per Transponder, um das Tor vor Ort und ohne Smartphone öffnen zu können, falls der Akku leer ist oder man anderen Personengruppen (evtl. temporären) Zutritt erteilen möchte

\section{Technische Umsetzung}

\subsection{Hardware}

\subsection{Implementierung Alexa}
\subsection{Datenbeschaffung}

\subsection{Modelltraining und Implementierung}

\subsection{Geometrische Lösung}
hier auf jeden noch die quelle 

\subsection{OCR}

\nocite{*}
